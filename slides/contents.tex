%%%%%%%%%%%%%%%%%%%%%%%%%%%%%%%%%%%%%%%%%%%%%%%%%%
\begin{frame}[fragile]{Soooo...}

\begin{itemize}

\item We want to build this nice webapp / distributed system / these microservices

\item We want to make sure ``Frontend'' and ``Backend'' play nicely together

\onslide<2->
\vspace{1.5em}
\item HOW???

\end{itemize}

\end{frame}

%%%%%%%%%%%%%%%%%%%%%%%%%%%%%%%%%%%%%%%%%%%%%%%%%%
\begin{frame}[fragile]{}

\begin{center}
{\Huge
``Super-Na\"ive'' Approach
}
\end{center}

\end{frame}

\begin{frame}[fragile]{}

\only<1>{
\includegraphics[width=\textwidth]{images/super-naive-approach-0.pdf}
}

\only<2>{
\includegraphics[width=\textwidth]{images/super-naive-approach-1.pdf}
}

\only<3>{
\includegraphics[width=\textwidth]{images/super-naive-approach-2.pdf}
}

\only<4>{
\includegraphics[width=\textwidth]{images/super-naive-approach-3.pdf}
}

\only<5>{
\includegraphics[width=\textwidth]{images/super-naive-approach-4.pdf}
}

\end{frame}

\begin{frame}[fragile]{}

\begin{center}
{\Huge
But $\ldots$
}
\end{center}

\end{frame}

\begin{frame}[fragile]{}

\only<1>{
\includegraphics[width=\textwidth]{images/super-naive-approach-4.pdf}
}

\only<2>{
\includegraphics[width=\textwidth]{images/super-naive-approach-5.pdf}
}

\only<3>{
\includegraphics[width=\textwidth]{images/super-naive-approach-6.pdf}
}

\only<4>{
\includegraphics[width=\textwidth]{images/super-naive-approach-7.pdf}
}

\end{frame}


%%%%%%%%%%%%%%%%%%%%%%%%%%%%%%%%%%%%%%%%%%%%%%%%%%
\begin{frame}[fragile]{}

\begin{center}
{\Huge
``Still Quite Na\"ive'' Approach
}
\end{center}

\end{frame}

\begin{frame}[fragile]{}

\only<1>{
\includegraphics[width=\textwidth]{images/still-quite-naive-approach-0.pdf}
}

\only<2>{
\includegraphics[width=\textwidth]{images/still-quite-naive-approach-1.pdf}
}

\only<3>{
\includegraphics[width=\textwidth]{images/still-quite-naive-approach-2.pdf}
}

\only<4>{
\includegraphics[width=\textwidth]{images/still-quite-naive-approach-3.pdf}
}

\only<5>{
\includegraphics[width=\textwidth]{images/still-quite-naive-approach-4.pdf}
}

\end{frame}

\begin{frame}[fragile]{}

\begin{center}
{\Huge
But $\ldots$
}
\end{center}

\end{frame}

\begin{frame}[fragile]{}

\only<1>{
\includegraphics[width=\textwidth]{images/still-quite-naive-approach-4.pdf}
}

\only<2>{
\includegraphics[width=\textwidth]{images/still-quite-naive-approach-5.pdf}
}

\only<3>{
\includegraphics[width=\textwidth]{images/still-quite-naive-approach-6.pdf}
}

\only<4>{
\includegraphics[width=\textwidth]{images/still-quite-naive-approach-7.pdf}
}

\end{frame}

%%%%%%%%%%%%%%%%%%%%%%%%%%%%%%%%%%%%%%%%%%%%%%%%%%
\begin{frame}[fragile]{}

\begin{center}
{\Huge
``Industry-Strength'' Approach: \\[1em]
Consumer-Driven Contract Testing
}
\end{center}

\end{frame}

\begin{frame}[fragile]{}

\only<1>{
\includegraphics[width=\textwidth]{images/CDCT1.pdf}
}

\only<2>{
\includegraphics[width=\textwidth]{images/CDCT2.pdf}
}

\only<3>{
\includegraphics[width=\textwidth]{images/CDCT3.pdf}
}

\only<4>{
\includegraphics[width=\textwidth]{images/CDCT4.pdf}
}

\only<5>{
\includegraphics[width=\textwidth]{images/CDCT5.pdf}
}

\only<6>{
\includegraphics[width=\textwidth]{images/CDCT6.pdf}
}

\only<7>{
\includegraphics[width=\textwidth]{images/CDCT7.pdf}
}

\end{frame}

%%%%%%%%%%%%%%%%%%%%%%%%%%%%%%%%%%%%%%%%%%%%%%%%%%
\begin{frame}[fragile]{}

\only<1>{
\includegraphics[height=\textheight]{images/petstore.png}
}
\only<2>{
\includegraphics[height=\textheight]{images/petstore-get.pdf}
}
\only<3>{
\includegraphics[height=\textheight]{images/petstore-add.pdf}
}
\only<4>{
\includegraphics[height=\textheight]{images/petstore-remove.pdf}
}


\end{frame}

%%%%%%%%%%%%%%%%%%%%%%%%%%%%%%%%%%%%%%%%%%%%%%%%%%
\begin{frame}[fragile]{Example: Pet Store Contracts I}

  \lstinputlisting{contracts/nopets}

\end{frame}


\begin{frame}[fragile]{Situation}

\begin{itemize}[<+->]
\item GET-Request
\item Does not depend on state
\item Easy to handle with CDCT
\end{itemize}
\end{frame}

\begin{frame}[fragile]{Questions}

\begin{itemize}[<+->]
\item Completeness
\begin{itemize}
\item Did we really capture all requests (+ responses) in our contract?
\end{itemize}
\end{itemize}

\end{frame}


%%%%%%%%%%%%%%%%%%%%%%%%%%%%%%%%%%%%%%%%%%%%%%%%%%
\begin{frame}[fragile]{Example: Pet Store Contracts II}

  \lstinputlisting{contracts/somepets}

\end{frame}


\begin{frame}[fragile]{Situation}

\begin{itemize}[<+->]
\item GET-Request that depends on state
\item POST-/PUT-/DELETE-Requests
\item Difficult to handle with CDCT
\item Backend state needs to be established somehow
\item State checks need to be established somehow
\end{itemize}
\end{frame}


\begin{frame}[fragile]{Questions}

\begin{itemize}[<+->]
\item Completeness
\begin{itemize}
\item Did we really capture all requests (+ responses) in our contract?
\item All possible combinations with different backend states?
\end{itemize}

\vspace{1em}
\item State Validity
\begin{itemize}
\item What are valid states in our stub?
\item Did we always establish a valid state in our stub?
\item How do we establish them (technically) before the request?
\item How do we validate them after the (modifying) request?
\end{itemize}

\vspace{1em}
\item Maintenance
\begin{itemize}
\item How can we keep track of our contracts and avoid redundancies?
\item How can we effectively maintain the contracts in case of changes?
\end{itemize}
\end{itemize}

\end{frame}

%%%%%%%%%%%%%%%%%%%%%%%%%%%%%%%%%%%%%%%%%%%%%%%%%%
\begin{frame}[fragile]{}

\begin{center}
{\Huge
We need the Functional Essence!
}
\end{center}

\end{frame}

\begin{frame}[fragile]{}
\vspace{-.65em}
\begin{center}
\includegraphics[height=\paperheight]{images/essence_large}
\end{center}

\end{frame}

%%%%%%%%%%%%%%%%%%%%%%%%%%%%%%%%%%%%%%%%%%%%%%%%%%
\begin{frame}[fragile]{}

\begin{center}
{\Huge
Sounds cool, but$\ldots$

\onslide<2->
\vspace{1.5em}
$\ldots$what does that mean?
}
\end{center}

\end{frame}

%%%%%%%%%%%%%%%%%%%%%%%%%%%%%%%%%%%%%%%%%%%%%%%%%%
\begin{frame}[fragile]{Functional Essence}

\begin{itemize}[<+->]
\item Fake Server
\item Lightweight Implementation

\vspace{1em}

\item Model

\vspace{1em}

\item Rapid Prototype
\item Minimal Viable Product
\end{itemize}

\end{frame}

%%%%%%%%%%%%%%%%%%%%%%%%%%%%%%%%%%%%%%%%%%%%%%%%%%
\begin{frame}[fragile]{Functional Essence Serves Many Purposes}

\begin{itemize}[<+->]
\item To learn about the domain
\item To discuss with domain experts
\item To validate assumptions at an early stage

\vspace{1em}

\item To check the real implementation:
\end{itemize}

\end{frame}

%%%%%%%%%%%%%%%%%%%%%%%%%%%%%%%%%%%%%%%%%%%%%%%%%%

% TODO Essence in Bilder einfügen

\begin{frame}[fragile]{}

\only<1>{
\includegraphics[width=\textwidth]{images/Essence-1.pdf}
}

\only<2>{
\includegraphics[width=\textwidth]{images/Essence-2.pdf}
}

\only<3>{
\includegraphics[width=\textwidth]{images/Essence-3.pdf}
}

\only<4>{
\includegraphics[width=\textwidth]{images/Essence-4.pdf}
}

\end{frame}


%%%%%%%%%%%%%%%%%%%%%%%%%%%%%%%%%%%%%%%%%%%%%%%%%%
\begin{frame}[fragile]{}

\begin{center}
{\Huge
Sounds cool, but$\ldots$

\onslide<2->
\vspace{1.5em}
$\ldots$how can I build this?
}
\end{center}

\end{frame}

%%%%%%%%%%%%%%%%%%%%%%%%%%%%%%%%%%%%%%%%%%%%%%%%%%
\begin{frame}[fragile]{How to Build a Functional Essence}

\begin{itemize}[<+->]
\item Implement the domain logic
\item In the simplest possible way
\item In an arbitrary language
\end{itemize}

\vspace{2em}

\onslide<+->
\begin{center}
\includegraphics[width=.3\textwidth]{images/essence_large.jpg}
\end{center}

\end{frame}

\begin{frame}[fragile]{The Pets Essence}

\begin{lstlisting}[language=JavaScript]
class Pets {
    constructor(){
        this._pets = [];
    }

    getPets() {
        return petSorter.sortPets(this._pets);
    }

    addPet(pet) {
        this._pets.push(pet);
        return 'Pet successfully added.';
    }

    removePet(pet) {
        this._pets = this._pets.filter(
            p => p.petName !== pet.petName || p.petType !== pet.petType);
        return 'Pet successfully removed.';
    }
}
\end{lstlisting}

\end{frame}

\begin{frame}[fragile]{The Overall Essence}

\begin{lstlisting}[language=Javascript]
class Essence {
    constructor() {
        this._pets = new Pets();
        this._somethingElse = new SomethingElse();
    }

    pets() {
        return this._pets;
    }
    
    somethingElse() {
        return this._somethingElse;
    }
}
\end{lstlisting}

\end{frame}

\begin{frame}[fragile]{The Essence App}

\begin{lstlisting}[language=Javascript]
let essence = new Essence();

router.get('/pets', (req, res) => {
    const pets = essence.pets().getPets();
    res.json({tag: 'Pets', pets});
});

router.post('/pets', (req, res) => {
    const message = essence.pets().addPet({ petName: req.body.petName, petPrice: req.body.petPrice, petType: req.body.petType });
    res.json({message});
});

router.delete('/pets', (req, res) => {
    const message = essence.pets().removePet({ petName: req.body.petName, petPrice: req.body.petPrice, petType: req.body.petType });
    res.json({message});
});
\end{lstlisting}

\end{frame}


\begin{frame}[fragile]{Important Addition}

\begin{lstlisting}[language=Javascript]
router.delete('/reset', (req, res) => {
    essence = new Essence();
    res.json({message: 'All pets successfully removed.'});
});
\end{lstlisting}

\end{frame}


%%%%%%%%%%%%%%%%%%%%%%%%%%%%%%%%%%%%%%%%%%%%%%%%%%
\begin{frame}[fragile]{Isn't That The Same Code As The Backend?}

\begin{itemize}
\onslide<2->
\item Maybe$\ldots$ but usually not:
\onslide<3->
\item Can be in another language
\onslide<4->
\item Can use simpler datatypes (e.g.~List vs.~HashMap)
\onslide<5->
\item No performance-tuning (e.g.~slower but simpler algorithms)
\onslide<6->
\item No technical overhead (e.g.~database)
\onslide<7->
\item Some parts can be left out
\onslide<8->
\item $\ldots$
\end{itemize}

\end{frame}

%%%%%%%%%%%%%%%%%%%%%%%%%%%%%%%%%%%%%%%%%%%%%%%%%%
\begin{frame}[fragile]{}

\begin{center}
{\Huge
How does the Comparison work?
}
\end{center}

\end{frame}

%%%%%%%%%%%%%%%%%%%%%%%%%%%%%%%%%%%%%%%%%%%%%%%%%%
\begin{frame}[fragile]{Detour: Quick Check / Property-Based Testing}

\onslide+<2->
\begin{itemize}
\item User specifies properties
\item Tool generates examples
\item Checks properties against examples
\end{itemize}

\onslide+<3->
\begin{lstlisting}
prop_RevRev xs = reverse (reverse xs) == xs
  where types = xs::[Int]
\end{lstlisting}

\onslide+<4->
\begin{lstlisting}
Main> quickCheck prop_RevRev
OK, passed 100 tests.
\end{lstlisting}

\onslide+<5->
\begin{lstlisting}
prop_RevId xs = reverse xs == xs
  where types = xs::[Int]
\end{lstlisting}

\onslide+<6->
\begin{lstlisting}
Main> quickCheck prop_RevId
Falsifiable, after 1 tests:
[-3,15]
\end{lstlisting}

\end{frame}

%%%%%%%%%%%%%%%%%%%%%%%%%%%%%%%%%%%%%%%%%%%%%%%%%%
\begin{frame}[fragile]{}

\begin{center}
{\Huge
How to Mimic QuickCheck?
}
\end{center}

\end{frame}


\begin{frame}[fragile]{}

\begin{center}
\only<1>{
\includegraphics[height=.9\textheight]{images/Comparator-1.pdf}
}

\only<2>{
\includegraphics[height=.9\textheight]{images/Comparator-2.pdf}
}

\only<3>{
\includegraphics[height=.9\textheight]{images/Comparator-3.pdf}
}

\only<4>{
\includegraphics[height=.9\textheight]{images/Comparator-4.pdf}
}

\only<5>{
\includegraphics[height=.9\textheight]{images/Comparator-5.pdf}
}

\only<6>{
\includegraphics[height=.9\textheight]{images/Comparator-6.pdf}
}

\only<7>{
\includegraphics[height=.9\textheight]{images/Comparator-7.pdf}
}

\only<8>{
\includegraphics[height=.9\textheight]{images/Comparator-8.pdf}
}
\end{center}

\end{frame}


%%%%%%%%%%%%%%%%%%%%%%%%%%%%%%%%%%%%%%%%%%%%%%%%%%
\begin{frame}[fragile]{}

\begin{center}
{\Huge
Comparator Implementation
}
\end{center}

\end{frame}

\begin{frame}[fragile]{Pet Generator}

\begin{lstlisting}[language=JavaScript]
const chooseFrom = arr => arr[Math.floor(arr.length * Math.random())];

const possibleNames = ['A', 'B', 'C'];
const possibleSpecies = ['Cat', 'Dog', 'Canary', 'Rabbit', 'Fish'];

const name = () => chooseFrom(possibleNames);
const price = () => Math.floor(150 * Math.random());
const species = () => chooseFrom(possibleSpecies);

const randomPet = () => 
          ({petName: name(), petPrice: price(), petType: species()});
\end{lstlisting}

\end{frame}

\begin{frame}[fragile]{Request Generator}

\begin{lstlisting}[language=JavaScript]
const resets = [
    // delete all pets
    () => ({url: '/reset', method: 'DELETE'}),
];

const modifyingRequestGenerator = [
    // addPet
    () => ({url: '/pets', method: 'POST', json: true, body: randomPet()}),
    // removePet
    () => ({url: '/pets', method: 'DELETE', json: true, body: randomPet()})
];

const comparisons = [
    // getPets
    () => ({url: '/pets', method: 'GET'}),
];
\end{lstlisting}

\end{frame}

\begin{frame}[fragile]{Query Submission}

\begin{lstlisting}[language=JavaScript]
const backend = {baseURL: 'http://localhost:9090'};
const essence = {baseURL: 'http://localhost:8080'};

const merge = (req, server) => 
        Object.assign({}, req, {url: server.baseURL + req.url});

const requestFunctionFor = (req, server) =>
    callback => request(merge(req, server), 
                        (err, response) => callback(err, response.body));

const runRequest = (req, callback) => {
    console.log('Now checking:', req);
    async.parallel({
        backend: requestFunctionFor(req, backend),
        essence: requestFunctionFor(req, essence)
    }, callback);
};
\end{lstlisting}

\end{frame}

\begin{frame}[fragile]{Request-Compare-Loop}

\begin{lstlisting}[language=JavaScript]
const requestAndCompare = (request, mainCallback) => {

    console.log('Running the modification request:');

    runRequest(request, (err, result) => {
        const backendString = JSON.stringify(result.backend);
        const essenceString = JSON.stringify(result.essence);
        
        if (backendString !== essenceString) {
            mainCallback('Backend and Essence responses differ! Backend: ' + backendString + ' - Essence: ' + essenceString);
            
        } else {
            
            console.log('Comparing all data:');
            compareEverything(mainCallback);
        }
    });
};
\end{lstlisting}

\end{frame}


\begin{frame}[fragile]{Comparisons}

\begin{lstlisting}[language=JavaScript]
function compareEverything(mainCallback) {
    async.map(comparisons, (itemFunc, callback) => 
        runRequest(itemFunc(), (err, res) => {
            if (res.backend === res.essence) {
                callback(null, null); // no differences
            } else {
                const formatDiff = formatter.formatDiff({
                    backend: JSON.parse(res.backend),
                    essence: JSON.parse(res.essence)
                });
                callback(null, formatDiff);
            }
        }),
        (err, results) => {
            const nonmatching = results.filter(x => x);
            mainCallback(nonmatching.length
                    ? 'Backend and Essence differ in their data' : null);
        });
}
\end{lstlisting}

\end{frame}



\begin{frame}[fragile]{The Main Loop}

\begin{lstlisting}[language=JavaScript]
const requests = [resets[0]()]; // initial reset

let count = 0;

while (count < 50) {
    count++;
    requests.push(chooseFrom(modifyingRequestGenerator)());
}

async.mapSeries(requests, requestAndCompare, (err, results) => {
    results.map(result => console.log(result));
    if (err) {
        console.log(err);
    }
});
\end{lstlisting}

\end{frame}


\begin{frame}[fragile]{Questions -- and Answers}

\begin{itemize}
\item Completeness
\begin{itemize}
\item Did we really capture all requests (+ responses) in our contract?
\item All possible combinations with different backend states?
\end{itemize}
\end{itemize}

\onslide<+->
\begin{itemize}[<+->]
\item CDCT:
\begin{itemize}
\item We must write all tests ourselves
\item We are responsible for completeness
\end{itemize}
\vspace{.4em}
\item Beyond CDCT:
\begin{itemize}
\item QuickCheck approach: We don't write tests
\item We only specify the possible routes
\item Completeness is established over time
\end{itemize}
\end{itemize}

\end{frame}


\begin{frame}[fragile]{Questions -- and Answers}

\begin{itemize}
\item State Validity
\begin{itemize}
\item What are valid states in our stub?
\item Did we always establish a valid state in our stub?
\item How do we establish them (technically) before the request?
\item How do we validate them after the (modifying) request?
\end{itemize}
\end{itemize}

\onslide<+->
\begin{itemize}[<+->]
\item CDCT:
\begin{itemize}
\item We establish the state ourselves for each test
\item We are responsible for the state's correctness
\end{itemize}
\vspace{.4em}
\item Beyond CDCT:
\begin{itemize}
\item The state is established through API calls
\item The state is always reachable and correct$^{*}$
\end{itemize}
\end{itemize}

\onslide<+->
 $^{*}$Assuming no coding errors!
 
\end{frame}


\begin{frame}[fragile]{Questions -- and Answers}

\begin{itemize}
\item Maintenance
\begin{itemize}
\item How can we keep track of our contracts and avoid redundancies?
\item How can we effectively maintain the contracts in case of changes?
\end{itemize}
\end{itemize}

\onslide<+->
\begin{itemize}[<+->]
\item CDCT:
\begin{itemize}
\item As difficult as maintaining normal tests
\item Code coverage helps identify uncovered areas
\end{itemize}
\vspace{.4em}
\item Beyond CDCT:
\begin{itemize}
\item We need to make sure that the driver knows all routes
\item Code coverage helps identify missing routes
\item Missing routes in Essence will be discovered because of errors
\item Missing logic in the Essence will be discovered
\begin{itemize}
\item initially because of unexpected behaviour when prototyping
\item later because of comparison errors
\end{itemize}
\end{itemize}
\end{itemize}

\end{frame}


%%%%%%%%%%%%%%%%%%%%%%%%%%%%%%%%%%%%%%%%%%%%%%%%%%
\begin{frame}[fragile]{A Word of Warning}

\begin{itemize}[<+->]
\item This is not an easy solution. The world remains complicated.
\item This is not a golden hammer. Don't apply it everywhere.
\item This is rather at the top of the test pyramid.
\end{itemize}

\end{frame}


%%%%%%%%%%%%%%%%%%%%%%%%%%%%%%%%%%%%%%%%%%%%%%%%%%
\begin{frame}{Thank you very much!}

  ~\\[1em]
  \begin{block}{Nicole Rauch}
    \begin{description}[Twitterxx]
    \item[E-Mail]  \href{mailto:info@nicole-rauch.de}{\texttt{info@nicole-rauch.de}}
    \item[Twitter] \href{http://twitter.com/NicoleRauch}{\texttt{@NicoleRauch}}
    \item[Web] \href{http://www.nicole-rauch.de}{\texttt{http://www.nicole-rauch.de}}
    \end{description}
  \end{block}

\begin{center}
EventStorming $\cdot$ Domain-Driven Design \\
Training $\cdot$ Coaching $\cdot$ Facilitation \\
Software Craftsmanship \\ 
React.js and Redux  \\
Functional Programming
\end{center}  

\end{frame}


%%%%%%%%%%%%%%%%%%%%%%%%%%%%%%%%%%%%%%%%%%%%%%%%%%
\begin{frame}{Credits}

Essence: Mark Morgan - Perfume
{\footnotesize \url{https://www.flickr.com/photos/markmorgantrinidad/5959461561}}


\end{frame}
